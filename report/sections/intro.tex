\documentclass[../main.tex]{subfiles}
\graphicspath{{\subfix{../images/}}}

\begin{document}

\subsection{Σκοπός Εφαρμογής}
Σκοπός της GoT-DB είναι η δημιουργία μίας βάσης δεδομένων που θα περιέχει όλες
τις πληροφορίες σε επίπεδο λεπτομέρειας σχετικά με τον κόσμο του Game of
Thrones. Στόχος είναι η εύκολη αναζήτηση αλλά και ποικιλία στον τρόπου που θα
μπορούν οι οπαδοί της σειράς αλλά και των βιβλίων να φιλτράρουν τα δεδομένα
ώστε να βλέπουν ακριβώς αυτό που τους ενδιαφέρει.

\subsection{Περιγραφή Εφαρμογής}
Στην GoT-DB τα δεδομένα που θα αποθηκεύονται είναι χαρακτήρες, περιοχές,
θρησκείες και πολλά άλλα καθώς ιδανικά θα θέλαμε να έχουμε στην βάση μας
οτιδήποτε μπορεί να θελήσει ένας χρήστης να αναζητήσει για την αγαπημένη του
σειρά. Οι χρήστες θα έχουν την δυνατότητα μετά από εγγραφή να κάνουν πρόταση
για προσθήκη νέας πληροφορίας.

\subsection{Απαιτήσεις Εφαρμογής σε Δεδομένα}

Ακριβείς εκτίμηση για το μέγεθος της βάσης δεν μπορεί να γίνει παρόλο που η
σειρά έχει ολοκληρωθεί, κι αυτό επειδή οι οπαδοί της ανακαλύπτουν νέες
λεπτομέρειες ακόμα κι σήμερα. Έχουμε όμως κάποια δεδομένα όπως: 

\begin{itemize}
  \item 73 αριθμός των επεισοδίων  
  \item 389 αριθμός χαρακτήρων που περιλαμβάνει και ανώνυμους χαρακτήρες (που
    όμως έπαιξαν κάποιον ρόλο στην πλοκή)
  \item 120 αριθμός τοποθεσιών που είτε διαδραματίστηκε κάποια σκηνή είτε απλά
    έγινε αναφορά από κάποιον χαρακτήρα
\end{itemize}

\end{document}
